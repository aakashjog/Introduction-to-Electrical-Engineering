\documentclass[fleqn, a4paper, 12pt, twoside]{article}
\usepackage{exsheets}
\usepackage{amsmath, amssymb, amsthm} %standard AMS packages
\usepackage{mathtools}
\usepackage{marginnote} %marginnotes
\usepackage{gensymb} %miscellaneous symbols
\usepackage{commath} %differential symbols
\usepackage{xcolor} %colours
\usepackage{cancel} %cancelling terms
\usepackage{siunitx} %formatting units
\usepackage{tikz, pgfplots} %diagrams
	\usetikzlibrary{calc, hobby, patterns, intersections}
\usepackage{graphicx} %inserting graphics
\usepackage{hyperref} %hyperlinks
\usepackage{datetime} %date and time
\usepackage{ulem} %underline for \emph{}
\usepackage{xfrac} %inline fractions
\usepackage{enumerate} %numbered lists
\usepackage{float} %inserting floats
\usepackage[american voltages]{circuitikz} %circuit diagrams
\usepackage{titlesec}

\newcommand\numberthis{\addtocounter{equation}{1}\tag{\theequation}} %adds numbers to specific equations in non-numbered list of equations

\newcommand{\AxisRotator}[1][rotate=0]{
	\tikz [x=0.25cm,y=0.60cm,line width=.2ex,-stealth,#1] \draw (0,0) arc (-150:150:1 and 1);%
} %rotation symbols on axes

\theoremstyle{definition}
\newtheorem{example}{Example}
\newtheorem{definition}{Definition}

\theoremstyle{theorem}
\newtheorem{theorem}{Theorem}

\newcommand{\curl}{\mathrm{curl\,}}

\makeatletter
\@addtoreset{section}{part} %resets section numbers in new part
\makeatother

\newcommand{\partbreak}{\clearpage} %starts every part on a new page

\newcommand\blfootnote[1]{%
	\begingroup
	\renewcommand\thefootnote{}\footnote{#1}%
	\addtocounter{footnote}{-1}%
	\endgroup
}

\SetupExSheets{solution/print = true}

%opening
\title{Introduction to Electrical Engineering}
\author{Aakash Jog}
\date{2014-15}

\begin{document}

\maketitle
%\setlength{\mathindent}{0pt}

\tableofcontents

 \newpage
%\part{General Information}

\section{Lecturer Information}

\textbf{Prof. Moshe Tur}\\
~\\
Office: Wolfson 413\\
Telephone: 03-640-8125\\
E-mail: tur@post.tau.ac.il\\

\section{Required Reading}

C.A. Desoer and E.S. Kuh: \textit{Basic Circuit Theory}, Mc-Graw-Hill, International Edition.

\newpage
\part{Basic Definitions and Laws}

\section{Basic Definitions}

\begin{definition}[Electrical circuit]
	A collection of interconnected components.
\end{definition}

\begin{definition}[Lumped component]
	An electrical component whose dimensions are very very small compared to the wavelength of the electromagnetic waves passing through it is called a lumped component.
\end{definition}

\begin{definition}[One port device]
	An electrical component with two terminals is called a one port device.
	\begin{figure}[H]
		\begin{circuitikz}
			\draw (0,0) -- (1,0) to [generic] (2,0) -- (3,0);
		\end{circuitikz}
	\end{figure}
\end{definition}

\begin{definition}[Nodes and branches]
	In the figure, all the black dots are called nodes. The parts of the circuit between two nodes are called branches.
	\begin{figure}[H]
		\begin{circuitikz}[scale = 0.8]
			\draw (0,0) to [generic, *-*, i<=$i_2$] (5,5) to [generic, *-*, i<=$i_1$] (10,0);
			\draw (0,0) to [generic, *-*, i<=$i_3$] (10,0);
			\draw (0,0) to [generic, *-*, i<=$i_5$] (5,-5) to [generic, *-*, i>=$i_2$] (10,0);
		\end{circuitikz}
	\end{figure}
\end{definition}

\section{Kirchoff's Laws}

\subsection{Kirchoff's Current Law}

The sum of all currents entering or exiting a node is zero.
\begin{figure}[H]
	\begin{circuitikz}[scale = 0.8]
		\draw (0,0) to [generic, *-*, i<=$i_2$] (5,5) to [generic, *-*, i<=$i_1$] (10,0);
		\draw (0,0) to [generic, *-*, i<=$i_3$] (10,0);
		\draw (0,0) to [generic, *-*, i<=$i_5$] (5,-5) to [generic, *-*, i>=$i_2$] (10,0);
	\end{circuitikz}
\end{figure}
	
\begin{equation*}
	i_1 + i_3 - i_4 = 0
\end{equation*}

\subsection{Kirchoff's Voltage Law}

The sum of all branch voltages along a closed loop is zero.
\begin{figure}[H]
	\begin{circuitikz}[american voltages]
		\draw (0,0) to [generic = 1, v<=$v_1$] (0,5) to [generic = 2, v<=$v_2$] (0,10) -- (5,10) to [generic = 3, v<=$v_3$] (5,5) to [generic = 5, v>=$v_5$] (5,0) -- (0,0);
		\draw (0,5) to [generic = 4, v>=$v_4$] (5,5);
	\end{circuitikz}
\end{figure}

\begin{align*}
	v_1 - v_4 - v_5 &= 0\\
	v_2 + v_3 + v_4 &= 0\\
	v_1 + v_2 + v_3 - v_5 &= 0
\end{align*}

\section{Components}

\subsection{Resistors}

\begin{definition}[Resistor]
	A two terminal component is called a resistor if the voltage across it at any given time $t$ is a function of the current at the same time $t$.
\end{definition}

\subsubsection{Linear Time Independent Resistor}

\begin{figure}[H]
	\begin{circuitikz}[american voltages]
		\draw (0,0) to [R, v<=$v$, i>=$i$] (5,0);
	\end{circuitikz}
\end{figure}

\begin{figure}[H]
	\begin{tikzpicture}[scale = 0.5]
		\begin{scope}[stealth-stealth]
			\draw (-5,0) -- (5,0) node [right] {$i$};
			\draw (0,-5) -- (0,5) node [above] {$v$};
		\end{scope}
		
		\draw (-4,-4) -- (4,4);
	\end{tikzpicture}
\end{figure}

\begin{align*}
	v(t) &= R \cdot i(t)\\
	i(t) &= G \cdot v(t)
\end{align*}
$R$ is called the resistance and $G$ is called the conductance.

\subsubsection{Non-linear Resistors (Diodes)}

\begin{figure}[H]
	\begin{circuitikz}[american voltages]
		\draw (0,0) to [Do, v<=$v$, i>=$i$] (5,0);
	\end{circuitikz}
\end{figure}

\begin{align*}
	i(t) &= I_s \left( e^{\dfrac{q \cdot v(t)}{k T}} - 1 \right)
\end{align*}

\begin{align*}
	I_s &= \textnormal{reverse current}\\
	k &= \textnormal{Boltzman constant}\\
	T &= \textnormal{$a$bsolute temperature}\\
	q &= \textnormal{electronic change}\\
	\dfrac{k T}{q} &= 0.026 (\textnormal{ at } 300 \si{\kelvin})
\end{align*}

\begin{figure}[H]
	\begin{tikzpicture}[scale = 0.5]
		\def\reverseCurrent{-0.2};
	
		\begin{scope}[stealth-stealth]
			\draw (-5,0) -- (5,0) node [right] {$v$};
			\draw (0,-5) -- (0,5) node [above] {$i$};
		\end{scope}
			
		\draw (-4,\reverseCurrent) to [out = 0, in = 190] (0,0) to [out = 10, in = 260] (2,4);
	\end{tikzpicture}
\end{figure}

\subsection{Independent Sources}

\subsubsection{Voltage Sources}

\begin{definition}[Voltage source]
	A two terminal component is called a voltage source if the voltage on its terminals is independent of the current through it.
	
	\begin{figure}[H]
		\begin{circuitikz}
			\draw (0,0) to [american voltage source, v = $v_s(t)$, i>=$i$] (0,5) to (5,5) to [generic] (5,0) to (0,0);
		\end{circuitikz}
	\end{figure}
	
	\begin{figure}[H]
		\begin{tikzpicture}[scale = 0.5]
			\def\i{2};
		
			\begin{scope}[stealth-stealth]
				\draw (-5,0) -- (5,0) node [right] {$i$};
				\draw (0,-5) -- (0,5) node [above] {$v$};
			\end{scope}
			
			\draw (-4,\i) -- (4,\i);
		\end{tikzpicture}
	\end{figure}
		
\end{definition}

\subsubsection{Current Sources}

\begin{definition}[Current source]
	A two terminal component is called a current source if it can supply a current $i_s(t)$ independent of the voltage across its terminals.

	\begin{figure}[H]
		\begin{circuitikz}
			\draw (0,0) to [american current source, i = $i_s(t)$] (0,5) to (5,5) to [generic] (5,0) to (0,0);
		\end{circuitikz}
	\end{figure}

	\begin{figure}[H]
		\begin{tikzpicture}[scale = 0.5]
			\def\v{2};
			
			\begin{scope}[stealth-stealth]
				\draw (-5,0) -- (5,0) node [right] {$i$};
				\draw (0,-5) -- (0,5) node [above] {$v$};
			\end{scope}
			
			\draw (\v,-4) -- (\v,4);
		\end{tikzpicture}
	\end{figure}
\end{definition}

\subsubsection{Real Batteries}

\begin{figure}[H]
	\begin{tikzpicture}[scale = 0.5]
		\def\v0{2};
		
		\begin{scope}[stealth-stealth]
			\draw (-5,0) -- (5,0) node [right] {$i$};
			\draw (0,-5) -- (0,5) node [above] {$v$};
		\end{scope}
		
		\draw (0,\v0) -- (2*\v0,0);
	\end{tikzpicture}
\end{figure}

\begin{figure}[H]
	\begin{circuitikz}
		\draw (0,0) to [battery1 = $V_0$] (0,2) to [R = $R_s$] (0,4) to (4,4) to [generic] (4,0) to (0,0);
	\end{circuitikz}
\end{figure}

\begin{align*}
	0 &= -V_0 + v_R + v\\
	v &= V_0 - v_R\\
	\therefore v &= V_0 - R_s i
\end{align*}

\subsection{Capacitor}

\begin{definition}[Capacitor]
	A capacitor is a two terminal device where $V$ is a fucntion of $q$.
	\begin{figure}[H]
		\begin{circuitikz}
			\draw (0,0) to [C] (5,0);
		\end{circuitikz}
	\end{figure}
\end{definition}

\subsubsection{Linear Capacitors}

If the charges on the terminals of a capacitor are $+q$ and $-q$, and the potential difference across it is $v$, the ratio between $q$ and $v$ is said to be the capacitance.\\
The unit of capacitance is farad or $\si{\farad}$.
\begin{align*}
	q &= C v\\
	\therefore i &= C \dod{v}{t}\\
	\therefore v(t) &= v(t_0) + \dfrac{1}{C} \int\limits_{t_0}^{t} i(t) \dif t
\end{align*}
\begin{figure}[H]
	\begin{circuitikz}
		\draw (0,0) to [C, v = $v$] (5,0);
	\end{circuitikz}
\end{figure}

\subsection{Inductor}

\begin{definition}[Inductor]
	\begin{figure}[H]
		\begin{circuitikz}
			\draw (0,0) to [L, v = $v(t)$, i> = $i(t)$] (5,0);
		\end{circuitikz}
	\end{figure}
	\begin{align*}
		v(t) &= \dod{\varphi}{t}\\
		\therefore v(t) &= L \dod{i}{t}\\
		\therefore i(t) &= i(t_0) + \dfrac{1}{L} \int\limits_{t_0}^{t} v(t) \dif t
	\end{align*}
\end{definition}

\section{Waveforms}

\subsection{DC (Constant Function)}

\begin{figure}[H]
	\begin{tikzpicture}[scale = 0.5]		
		\begin{scope}[stealth-stealth]
			\draw (-5,0) -- (5,0) node [right] {$t$};
			\draw (0,-5) -- (0,5);
		\end{scope}
		
		\draw (-4,3) -- (4,3);
		\end{tikzpicture}
\end{figure}

\subsection{Sinusoidal Wave}

\begin{figure}[H]
	\begin{tikzpicture}[scale = 0.5]		
		\begin{scope}[stealth-stealth]
			\draw (-5,0) -- (5,0) node [right] {$t$};
			\draw (0,-5) -- (0,5);
		\end{scope}
		
		\draw [domain = 0: 4] plot (\x, {(cos(2*\x r)});
	\end{tikzpicture}
\end{figure}

\subsection{Step Function}

\begin{figure}[H]
	\begin{tikzpicture}[scale = 0.5]	
		\begin{scope}[gray, stealth-stealth]
			\draw (-5,0) -- (5,0) node [right] {$t$};
			\draw (0,-5) -- (0,5);
		\end{scope}
		
		\draw (-4,0) -- (1,0) -- (1,2) -- (4,2);
		
		\node [below] at (1,0) {$\tau$};
	\end{tikzpicture}
\end{figure}

\begin{align*}
	u(t) &=
		\begin{cases}
			0 &;\quad t < 0\\
			\dfrac{1}{2} &;\quad t = \tau\\
			1 &;\quad t > \tau\\
		\end{cases}
\end{align*}

\subsection{Rectangular Pulse}

\begin{figure}[H]
	\begin{tikzpicture}[scale = 0.5]	
		\begin{scope}[gray, stealth-stealth]
			\draw (-5,0) -- (5,0) node [right] {$t$};
			\draw (0,-5) -- (0,5);
		\end{scope}
		
		\draw (-4,0) -- (0,0) -- (0,2) -- (2,2) -- (2,0) -- (4,0);
		
		\node [below] at (2,0) {$\Delta$};
	\end{tikzpicture}
\end{figure}

\begin{align*}
	P_{\Delta}(t) &= \dfrac{u(t) - u(t - \Delta)}{\Delta}
\end{align*}

\begin{align*}
	P_{\Delta}(t) &=
		\begin{cases}
			0 &;\quad t < 0\\
			\dfrac{1}{\Delta} &;\quad t = 0\\
			1 &;\quad t > 0\\
		\end{cases}
\end{align*}

\subsection{Dirac $\delta$ function}

\begin{align*}
	\delta(t) &= \lim\limits_{\Delta \to 0} P_{\Delta}(t)
\end{align*}

\begin{align*}
	S(\Delta) &= \int\limits_{-\infty}^{\infty} P_{\Delta}(t) f(t) \dif t\\
	\intertext{As $\Delta \to 0$,}
	S(\Delta) &= \int\limits_{-\infty}^{\infty} P_{\Delta}(t) f(0) \dif t\\
	&= f(0) \int\limits_{-\infty}^{\infty} P_0(t) \dif t\\
	&= f(0)
\end{align*}

\begin{align*}
	\delta(t) &=
		\begin{cases}
			0 &;\quad t \neq 0\\
			\infty &;\quad t = 0\\
		\end{cases}
\end{align*}

\begin{align*}
	\int\limits_{-\infty}^{\infty} \delta(t) f(t) \dif t &= f(0)\\
	\int\limits_{-\infty}^{\infty} \delta(t - \tau) f(t) \dif t &= f(\tau)\\
	\int\limits_{-\infty}^{\infty} \delta(a t) f(t) \dif t &= \dfrac{1}{|a|} \delta(t)
\end{align*}

\subsection{Ramp Function}

\begin{figure}[H]
	\begin{tikzpicture}[scale = 0.5]	
		\begin{scope}[gray, stealth-stealth]
			\draw (-5,0) -- (5,0) node [right] {$t$};
			\draw (0,-5) -- (0,5);
		\end{scope}
		
		\draw (-4,0) -- (0,0) -- (4,4);
	\end{tikzpicture}
\end{figure}

\begin{align*}
	r(t) &= t u(t)
\end{align*}

\subsection{Doublet Function}

\begin{align*}
	\delta' (t) &= \dod{\delta(t)}{t}
\end{align*}

\subsection{Relation Between Standard Waveforms}

\begin{equation*}
	r(t) \xrightleftharpoons[\int_{-\infty}^{t}]{\od{}{t}} u(t) \xrightleftharpoons[\int_{-\infty}^{t}]{\od{}{t}} \delta(t) \xrightleftharpoons[\int_{-\infty}^{t}]{\od{}{t}} \delta' (t) 
\end{equation*}

\begin{question}
	Express the following wave as a sum of standard waveforms.
	\begin{figure}[H]
		\begin{tikzpicture}
			\begin{scope}[gray, stealth-stealth]
				\draw (-1,0) -- (5,0) node [right] {$t$};
				\draw (0,-1) -- (0,5);
			\end{scope}
			
			\draw (0,0) -- (1,0) -- (1,1) -- (2,1) -- (2,2) -- (4,2);
			
			\foreach \i in {1,...,4}
			{
				\node [below] at (\i,0) {$\i$};
			}
		\end{tikzpicture}
	\end{figure}
\end{question}

\begin{solution}
	\begin{equation*}
		f(t) = u(t - 1) + u(t - 2)
	\end{equation*}
\end{solution}

\begin{question}
	Express the following wave as a sum of standard waveforms.
	\begin{figure}[H]
		\begin{tikzpicture}
			\begin{scope}[gray, stealth-stealth]
				\draw (-1,0) -- (5,0) node [right] {$t$};
				\draw (0,-1) -- (0,5);
			\end{scope}
			
			\draw (0,0) -- (1,1) -- (4,1);
			
			\foreach \i in {1,...,4}
			{
				\node [below] at (\i,0) {$\i$};
			}
		\end{tikzpicture}
	\end{figure}
\end{question}

\begin{solution}
	\begin{equation*}
		f(t) = r(t) + r(t - 1)
	\end{equation*}
\end{solution}

\section{Power and Energy}

The instantaneous power supplied to a load is
\begin{align*}
	P(t) &= v(t) \cdot i(t)
\end{align*}
where $v(t)$ and $i(t)$ are in matched directions.

The energy supplied to a load from time $t_0$ to time $t$ is
\begin{align*}
	W(t_0, t) &= \int\limits_{t_0}^{t} P(t) \dif t\\
	&= \int\limits_{t_0}^{t} v(t) \cdot i(t) \dif t
\end{align*}

\subsection{Energy Stored in a Capacitor}

\begin{align*}
	W(t_0, t) &= \int\limits_{t_0}^{t} v(t) i(t) \dif t\\
	\intertext{As $i(t) = \dod{q}{t}$, $\dif q = i(t) \dif t$. Therefore,}
	W(t_0, t) &= \int\limits_{q(t_0)}^{q(t)} v(q) \dif q\\
	&= \int\limits_{q(t_0)}^{q(t)} \dfrac{q}{C} \dif q\\
	&= \dfrac{q^2}{2c}\\
	&= \dfrac{1}{2} c v^2
\end{align*}

\subsection{Energy Stored in an Inductor}

\begin{align*}
	W(t_0, t) &= \int\limits_{t_0}^{t} v(t) i(t) \dif t\\
	\intertext{As $v(t) = \dod{\varphi}{t}$, $\dif \varphi = v(t) \dif t$. Therefore,}
	W(t_0, t) &= \int\limits_{\varphi(t_0)}^{\varphi(t)} i(\varphi) \dif \varphi\\
	&= \dfrac{\varphi^2}{2 L}\\
	&= \dfrac{1}{2} L i^2
\end{align*}

\part{Simple Circuits}

\section{Equivalent Circuits}

\begin{definition}
	Two circuits are said to be equivalent if they have the same $v(t)$-$i(t)$ relationships.
\end{definition}

\begin{figure}[H]
	\begin{circuitikz}
		\draw (0,0) to [R = $R_1$] (2,0) to [R = $R_2$] (4,0); 
	\end{circuitikz}
\end{figure}

Let the voltage across $R_1$ be $v_1$ and across $R_2$ be $v_2$.\\
Let the current through $R_1$ be $i_1$ and through $R_2$ be $i_2$.\\
Therefore,
\begin{align*}
	v_1 &= f_1(i_1)\\
	v_2 &= f_2(i_2)
\end{align*}
Therefore,
\begin{align*}
	v &= v_1 + v_2\\
	&= f_1(i_1) + f_2(i_2)\\
	&= f_1(i) + f_2(i)\\
	&= f_3(i)
\end{align*}
Therefore, the system of resistors is equivalent to a single resistor $R_3$.
\end{document}